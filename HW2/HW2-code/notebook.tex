
% Default to the notebook output style

    


% Inherit from the specified cell style.




    
\documentclass[11pt]{article}

    
    
    \usepackage[T1]{fontenc}
    % Nicer default font (+ math font) than Computer Modern for most use cases
    \usepackage{mathpazo}

    % Basic figure setup, for now with no caption control since it's done
    % automatically by Pandoc (which extracts ![](path) syntax from Markdown).
    \usepackage{graphicx}
    % We will generate all images so they have a width \maxwidth. This means
    % that they will get their normal width if they fit onto the page, but
    % are scaled down if they would overflow the margins.
    \makeatletter
    \def\maxwidth{\ifdim\Gin@nat@width>\linewidth\linewidth
    \else\Gin@nat@width\fi}
    \makeatother
    \let\Oldincludegraphics\includegraphics
    % Set max figure width to be 80% of text width, for now hardcoded.
    \renewcommand{\includegraphics}[1]{\Oldincludegraphics[width=.8\maxwidth]{#1}}
    % Ensure that by default, figures have no caption (until we provide a
    % proper Figure object with a Caption API and a way to capture that
    % in the conversion process - todo).
    \usepackage{caption}
    \DeclareCaptionLabelFormat{nolabel}{}
    \captionsetup{labelformat=nolabel}

    \usepackage{adjustbox} % Used to constrain images to a maximum size 
    \usepackage{xcolor} % Allow colors to be defined
    \usepackage{enumerate} % Needed for markdown enumerations to work
    \usepackage{geometry} % Used to adjust the document margins
    \usepackage{amsmath} % Equations
    \usepackage{amssymb} % Equations
    \usepackage{textcomp} % defines textquotesingle
    % Hack from http://tex.stackexchange.com/a/47451/13684:
    \AtBeginDocument{%
        \def\PYZsq{\textquotesingle}% Upright quotes in Pygmentized code
    }
    \usepackage{upquote} % Upright quotes for verbatim code
    \usepackage{eurosym} % defines \euro
    \usepackage[mathletters]{ucs} % Extended unicode (utf-8) support
    \usepackage[utf8x]{inputenc} % Allow utf-8 characters in the tex document
    \usepackage{fancyvrb} % verbatim replacement that allows latex
    \usepackage{grffile} % extends the file name processing of package graphics 
                         % to support a larger range 
    % The hyperref package gives us a pdf with properly built
    % internal navigation ('pdf bookmarks' for the table of contents,
    % internal cross-reference links, web links for URLs, etc.)
    \usepackage{hyperref}
    \usepackage{longtable} % longtable support required by pandoc >1.10
    \usepackage{booktabs}  % table support for pandoc > 1.12.2
    \usepackage[inline]{enumitem} % IRkernel/repr support (it uses the enumerate* environment)
    \usepackage[normalem]{ulem} % ulem is needed to support strikethroughs (\sout)
                                % normalem makes italics be italics, not underlines
    

    
    
    % Colors for the hyperref package
    \definecolor{urlcolor}{rgb}{0,.145,.698}
    \definecolor{linkcolor}{rgb}{.71,0.21,0.01}
    \definecolor{citecolor}{rgb}{.12,.54,.11}

    % ANSI colors
    \definecolor{ansi-black}{HTML}{3E424D}
    \definecolor{ansi-black-intense}{HTML}{282C36}
    \definecolor{ansi-red}{HTML}{E75C58}
    \definecolor{ansi-red-intense}{HTML}{B22B31}
    \definecolor{ansi-green}{HTML}{00A250}
    \definecolor{ansi-green-intense}{HTML}{007427}
    \definecolor{ansi-yellow}{HTML}{DDB62B}
    \definecolor{ansi-yellow-intense}{HTML}{B27D12}
    \definecolor{ansi-blue}{HTML}{208FFB}
    \definecolor{ansi-blue-intense}{HTML}{0065CA}
    \definecolor{ansi-magenta}{HTML}{D160C4}
    \definecolor{ansi-magenta-intense}{HTML}{A03196}
    \definecolor{ansi-cyan}{HTML}{60C6C8}
    \definecolor{ansi-cyan-intense}{HTML}{258F8F}
    \definecolor{ansi-white}{HTML}{C5C1B4}
    \definecolor{ansi-white-intense}{HTML}{A1A6B2}

    % commands and environments needed by pandoc snippets
    % extracted from the output of `pandoc -s`
    \providecommand{\tightlist}{%
      \setlength{\itemsep}{0pt}\setlength{\parskip}{0pt}}
    \DefineVerbatimEnvironment{Highlighting}{Verbatim}{commandchars=\\\{\}}
    % Add ',fontsize=\small' for more characters per line
    \newenvironment{Shaded}{}{}
    \newcommand{\KeywordTok}[1]{\textcolor[rgb]{0.00,0.44,0.13}{\textbf{{#1}}}}
    \newcommand{\DataTypeTok}[1]{\textcolor[rgb]{0.56,0.13,0.00}{{#1}}}
    \newcommand{\DecValTok}[1]{\textcolor[rgb]{0.25,0.63,0.44}{{#1}}}
    \newcommand{\BaseNTok}[1]{\textcolor[rgb]{0.25,0.63,0.44}{{#1}}}
    \newcommand{\FloatTok}[1]{\textcolor[rgb]{0.25,0.63,0.44}{{#1}}}
    \newcommand{\CharTok}[1]{\textcolor[rgb]{0.25,0.44,0.63}{{#1}}}
    \newcommand{\StringTok}[1]{\textcolor[rgb]{0.25,0.44,0.63}{{#1}}}
    \newcommand{\CommentTok}[1]{\textcolor[rgb]{0.38,0.63,0.69}{\textit{{#1}}}}
    \newcommand{\OtherTok}[1]{\textcolor[rgb]{0.00,0.44,0.13}{{#1}}}
    \newcommand{\AlertTok}[1]{\textcolor[rgb]{1.00,0.00,0.00}{\textbf{{#1}}}}
    \newcommand{\FunctionTok}[1]{\textcolor[rgb]{0.02,0.16,0.49}{{#1}}}
    \newcommand{\RegionMarkerTok}[1]{{#1}}
    \newcommand{\ErrorTok}[1]{\textcolor[rgb]{1.00,0.00,0.00}{\textbf{{#1}}}}
    \newcommand{\NormalTok}[1]{{#1}}
    
    % Additional commands for more recent versions of Pandoc
    \newcommand{\ConstantTok}[1]{\textcolor[rgb]{0.53,0.00,0.00}{{#1}}}
    \newcommand{\SpecialCharTok}[1]{\textcolor[rgb]{0.25,0.44,0.63}{{#1}}}
    \newcommand{\VerbatimStringTok}[1]{\textcolor[rgb]{0.25,0.44,0.63}{{#1}}}
    \newcommand{\SpecialStringTok}[1]{\textcolor[rgb]{0.73,0.40,0.53}{{#1}}}
    \newcommand{\ImportTok}[1]{{#1}}
    \newcommand{\DocumentationTok}[1]{\textcolor[rgb]{0.73,0.13,0.13}{\textit{{#1}}}}
    \newcommand{\AnnotationTok}[1]{\textcolor[rgb]{0.38,0.63,0.69}{\textbf{\textit{{#1}}}}}
    \newcommand{\CommentVarTok}[1]{\textcolor[rgb]{0.38,0.63,0.69}{\textbf{\textit{{#1}}}}}
    \newcommand{\VariableTok}[1]{\textcolor[rgb]{0.10,0.09,0.49}{{#1}}}
    \newcommand{\ControlFlowTok}[1]{\textcolor[rgb]{0.00,0.44,0.13}{\textbf{{#1}}}}
    \newcommand{\OperatorTok}[1]{\textcolor[rgb]{0.40,0.40,0.40}{{#1}}}
    \newcommand{\BuiltInTok}[1]{{#1}}
    \newcommand{\ExtensionTok}[1]{{#1}}
    \newcommand{\PreprocessorTok}[1]{\textcolor[rgb]{0.74,0.48,0.00}{{#1}}}
    \newcommand{\AttributeTok}[1]{\textcolor[rgb]{0.49,0.56,0.16}{{#1}}}
    \newcommand{\InformationTok}[1]{\textcolor[rgb]{0.38,0.63,0.69}{\textbf{\textit{{#1}}}}}
    \newcommand{\WarningTok}[1]{\textcolor[rgb]{0.38,0.63,0.69}{\textbf{\textit{{#1}}}}}
    
    
    % Define a nice break command that doesn't care if a line doesn't already
    % exist.
    \def\br{\hspace*{\fill} \\* }
    % Math Jax compatability definitions
    \def\gt{>}
    \def\lt{<}
    % Document parameters
    \title{softmax}
    
    
    

    % Pygments definitions
    
\makeatletter
\def\PY@reset{\let\PY@it=\relax \let\PY@bf=\relax%
    \let\PY@ul=\relax \let\PY@tc=\relax%
    \let\PY@bc=\relax \let\PY@ff=\relax}
\def\PY@tok#1{\csname PY@tok@#1\endcsname}
\def\PY@toks#1+{\ifx\relax#1\empty\else%
    \PY@tok{#1}\expandafter\PY@toks\fi}
\def\PY@do#1{\PY@bc{\PY@tc{\PY@ul{%
    \PY@it{\PY@bf{\PY@ff{#1}}}}}}}
\def\PY#1#2{\PY@reset\PY@toks#1+\relax+\PY@do{#2}}

\expandafter\def\csname PY@tok@w\endcsname{\def\PY@tc##1{\textcolor[rgb]{0.73,0.73,0.73}{##1}}}
\expandafter\def\csname PY@tok@c\endcsname{\let\PY@it=\textit\def\PY@tc##1{\textcolor[rgb]{0.25,0.50,0.50}{##1}}}
\expandafter\def\csname PY@tok@cp\endcsname{\def\PY@tc##1{\textcolor[rgb]{0.74,0.48,0.00}{##1}}}
\expandafter\def\csname PY@tok@k\endcsname{\let\PY@bf=\textbf\def\PY@tc##1{\textcolor[rgb]{0.00,0.50,0.00}{##1}}}
\expandafter\def\csname PY@tok@kp\endcsname{\def\PY@tc##1{\textcolor[rgb]{0.00,0.50,0.00}{##1}}}
\expandafter\def\csname PY@tok@kt\endcsname{\def\PY@tc##1{\textcolor[rgb]{0.69,0.00,0.25}{##1}}}
\expandafter\def\csname PY@tok@o\endcsname{\def\PY@tc##1{\textcolor[rgb]{0.40,0.40,0.40}{##1}}}
\expandafter\def\csname PY@tok@ow\endcsname{\let\PY@bf=\textbf\def\PY@tc##1{\textcolor[rgb]{0.67,0.13,1.00}{##1}}}
\expandafter\def\csname PY@tok@nb\endcsname{\def\PY@tc##1{\textcolor[rgb]{0.00,0.50,0.00}{##1}}}
\expandafter\def\csname PY@tok@nf\endcsname{\def\PY@tc##1{\textcolor[rgb]{0.00,0.00,1.00}{##1}}}
\expandafter\def\csname PY@tok@nc\endcsname{\let\PY@bf=\textbf\def\PY@tc##1{\textcolor[rgb]{0.00,0.00,1.00}{##1}}}
\expandafter\def\csname PY@tok@nn\endcsname{\let\PY@bf=\textbf\def\PY@tc##1{\textcolor[rgb]{0.00,0.00,1.00}{##1}}}
\expandafter\def\csname PY@tok@ne\endcsname{\let\PY@bf=\textbf\def\PY@tc##1{\textcolor[rgb]{0.82,0.25,0.23}{##1}}}
\expandafter\def\csname PY@tok@nv\endcsname{\def\PY@tc##1{\textcolor[rgb]{0.10,0.09,0.49}{##1}}}
\expandafter\def\csname PY@tok@no\endcsname{\def\PY@tc##1{\textcolor[rgb]{0.53,0.00,0.00}{##1}}}
\expandafter\def\csname PY@tok@nl\endcsname{\def\PY@tc##1{\textcolor[rgb]{0.63,0.63,0.00}{##1}}}
\expandafter\def\csname PY@tok@ni\endcsname{\let\PY@bf=\textbf\def\PY@tc##1{\textcolor[rgb]{0.60,0.60,0.60}{##1}}}
\expandafter\def\csname PY@tok@na\endcsname{\def\PY@tc##1{\textcolor[rgb]{0.49,0.56,0.16}{##1}}}
\expandafter\def\csname PY@tok@nt\endcsname{\let\PY@bf=\textbf\def\PY@tc##1{\textcolor[rgb]{0.00,0.50,0.00}{##1}}}
\expandafter\def\csname PY@tok@nd\endcsname{\def\PY@tc##1{\textcolor[rgb]{0.67,0.13,1.00}{##1}}}
\expandafter\def\csname PY@tok@s\endcsname{\def\PY@tc##1{\textcolor[rgb]{0.73,0.13,0.13}{##1}}}
\expandafter\def\csname PY@tok@sd\endcsname{\let\PY@it=\textit\def\PY@tc##1{\textcolor[rgb]{0.73,0.13,0.13}{##1}}}
\expandafter\def\csname PY@tok@si\endcsname{\let\PY@bf=\textbf\def\PY@tc##1{\textcolor[rgb]{0.73,0.40,0.53}{##1}}}
\expandafter\def\csname PY@tok@se\endcsname{\let\PY@bf=\textbf\def\PY@tc##1{\textcolor[rgb]{0.73,0.40,0.13}{##1}}}
\expandafter\def\csname PY@tok@sr\endcsname{\def\PY@tc##1{\textcolor[rgb]{0.73,0.40,0.53}{##1}}}
\expandafter\def\csname PY@tok@ss\endcsname{\def\PY@tc##1{\textcolor[rgb]{0.10,0.09,0.49}{##1}}}
\expandafter\def\csname PY@tok@sx\endcsname{\def\PY@tc##1{\textcolor[rgb]{0.00,0.50,0.00}{##1}}}
\expandafter\def\csname PY@tok@m\endcsname{\def\PY@tc##1{\textcolor[rgb]{0.40,0.40,0.40}{##1}}}
\expandafter\def\csname PY@tok@gh\endcsname{\let\PY@bf=\textbf\def\PY@tc##1{\textcolor[rgb]{0.00,0.00,0.50}{##1}}}
\expandafter\def\csname PY@tok@gu\endcsname{\let\PY@bf=\textbf\def\PY@tc##1{\textcolor[rgb]{0.50,0.00,0.50}{##1}}}
\expandafter\def\csname PY@tok@gd\endcsname{\def\PY@tc##1{\textcolor[rgb]{0.63,0.00,0.00}{##1}}}
\expandafter\def\csname PY@tok@gi\endcsname{\def\PY@tc##1{\textcolor[rgb]{0.00,0.63,0.00}{##1}}}
\expandafter\def\csname PY@tok@gr\endcsname{\def\PY@tc##1{\textcolor[rgb]{1.00,0.00,0.00}{##1}}}
\expandafter\def\csname PY@tok@ge\endcsname{\let\PY@it=\textit}
\expandafter\def\csname PY@tok@gs\endcsname{\let\PY@bf=\textbf}
\expandafter\def\csname PY@tok@gp\endcsname{\let\PY@bf=\textbf\def\PY@tc##1{\textcolor[rgb]{0.00,0.00,0.50}{##1}}}
\expandafter\def\csname PY@tok@go\endcsname{\def\PY@tc##1{\textcolor[rgb]{0.53,0.53,0.53}{##1}}}
\expandafter\def\csname PY@tok@gt\endcsname{\def\PY@tc##1{\textcolor[rgb]{0.00,0.27,0.87}{##1}}}
\expandafter\def\csname PY@tok@err\endcsname{\def\PY@bc##1{\setlength{\fboxsep}{0pt}\fcolorbox[rgb]{1.00,0.00,0.00}{1,1,1}{\strut ##1}}}
\expandafter\def\csname PY@tok@kc\endcsname{\let\PY@bf=\textbf\def\PY@tc##1{\textcolor[rgb]{0.00,0.50,0.00}{##1}}}
\expandafter\def\csname PY@tok@kd\endcsname{\let\PY@bf=\textbf\def\PY@tc##1{\textcolor[rgb]{0.00,0.50,0.00}{##1}}}
\expandafter\def\csname PY@tok@kn\endcsname{\let\PY@bf=\textbf\def\PY@tc##1{\textcolor[rgb]{0.00,0.50,0.00}{##1}}}
\expandafter\def\csname PY@tok@kr\endcsname{\let\PY@bf=\textbf\def\PY@tc##1{\textcolor[rgb]{0.00,0.50,0.00}{##1}}}
\expandafter\def\csname PY@tok@bp\endcsname{\def\PY@tc##1{\textcolor[rgb]{0.00,0.50,0.00}{##1}}}
\expandafter\def\csname PY@tok@fm\endcsname{\def\PY@tc##1{\textcolor[rgb]{0.00,0.00,1.00}{##1}}}
\expandafter\def\csname PY@tok@vc\endcsname{\def\PY@tc##1{\textcolor[rgb]{0.10,0.09,0.49}{##1}}}
\expandafter\def\csname PY@tok@vg\endcsname{\def\PY@tc##1{\textcolor[rgb]{0.10,0.09,0.49}{##1}}}
\expandafter\def\csname PY@tok@vi\endcsname{\def\PY@tc##1{\textcolor[rgb]{0.10,0.09,0.49}{##1}}}
\expandafter\def\csname PY@tok@vm\endcsname{\def\PY@tc##1{\textcolor[rgb]{0.10,0.09,0.49}{##1}}}
\expandafter\def\csname PY@tok@sa\endcsname{\def\PY@tc##1{\textcolor[rgb]{0.73,0.13,0.13}{##1}}}
\expandafter\def\csname PY@tok@sb\endcsname{\def\PY@tc##1{\textcolor[rgb]{0.73,0.13,0.13}{##1}}}
\expandafter\def\csname PY@tok@sc\endcsname{\def\PY@tc##1{\textcolor[rgb]{0.73,0.13,0.13}{##1}}}
\expandafter\def\csname PY@tok@dl\endcsname{\def\PY@tc##1{\textcolor[rgb]{0.73,0.13,0.13}{##1}}}
\expandafter\def\csname PY@tok@s2\endcsname{\def\PY@tc##1{\textcolor[rgb]{0.73,0.13,0.13}{##1}}}
\expandafter\def\csname PY@tok@sh\endcsname{\def\PY@tc##1{\textcolor[rgb]{0.73,0.13,0.13}{##1}}}
\expandafter\def\csname PY@tok@s1\endcsname{\def\PY@tc##1{\textcolor[rgb]{0.73,0.13,0.13}{##1}}}
\expandafter\def\csname PY@tok@mb\endcsname{\def\PY@tc##1{\textcolor[rgb]{0.40,0.40,0.40}{##1}}}
\expandafter\def\csname PY@tok@mf\endcsname{\def\PY@tc##1{\textcolor[rgb]{0.40,0.40,0.40}{##1}}}
\expandafter\def\csname PY@tok@mh\endcsname{\def\PY@tc##1{\textcolor[rgb]{0.40,0.40,0.40}{##1}}}
\expandafter\def\csname PY@tok@mi\endcsname{\def\PY@tc##1{\textcolor[rgb]{0.40,0.40,0.40}{##1}}}
\expandafter\def\csname PY@tok@il\endcsname{\def\PY@tc##1{\textcolor[rgb]{0.40,0.40,0.40}{##1}}}
\expandafter\def\csname PY@tok@mo\endcsname{\def\PY@tc##1{\textcolor[rgb]{0.40,0.40,0.40}{##1}}}
\expandafter\def\csname PY@tok@ch\endcsname{\let\PY@it=\textit\def\PY@tc##1{\textcolor[rgb]{0.25,0.50,0.50}{##1}}}
\expandafter\def\csname PY@tok@cm\endcsname{\let\PY@it=\textit\def\PY@tc##1{\textcolor[rgb]{0.25,0.50,0.50}{##1}}}
\expandafter\def\csname PY@tok@cpf\endcsname{\let\PY@it=\textit\def\PY@tc##1{\textcolor[rgb]{0.25,0.50,0.50}{##1}}}
\expandafter\def\csname PY@tok@c1\endcsname{\let\PY@it=\textit\def\PY@tc##1{\textcolor[rgb]{0.25,0.50,0.50}{##1}}}
\expandafter\def\csname PY@tok@cs\endcsname{\let\PY@it=\textit\def\PY@tc##1{\textcolor[rgb]{0.25,0.50,0.50}{##1}}}

\def\PYZbs{\char`\\}
\def\PYZus{\char`\_}
\def\PYZob{\char`\{}
\def\PYZcb{\char`\}}
\def\PYZca{\char`\^}
\def\PYZam{\char`\&}
\def\PYZlt{\char`\<}
\def\PYZgt{\char`\>}
\def\PYZsh{\char`\#}
\def\PYZpc{\char`\%}
\def\PYZdl{\char`\$}
\def\PYZhy{\char`\-}
\def\PYZsq{\char`\'}
\def\PYZdq{\char`\"}
\def\PYZti{\char`\~}
% for compatibility with earlier versions
\def\PYZat{@}
\def\PYZlb{[}
\def\PYZrb{]}
\makeatother


    % Exact colors from NB
    \definecolor{incolor}{rgb}{0.0, 0.0, 0.5}
    \definecolor{outcolor}{rgb}{0.545, 0.0, 0.0}



    
    % Prevent overflowing lines due to hard-to-break entities
    \sloppy 
    % Setup hyperref package
    \hypersetup{
      breaklinks=true,  % so long urls are correctly broken across lines
      colorlinks=true,
      urlcolor=urlcolor,
      linkcolor=linkcolor,
      citecolor=citecolor,
      }
    % Slightly bigger margins than the latex defaults
    
    \geometry{verbose,tmargin=1in,bmargin=1in,lmargin=1in,rmargin=1in}
    
    

    \begin{document}
    
    
    \maketitle
    
    

    
    \hypertarget{this-is-the-softmax-workbook-for-ece-c147c247-assignment-2}{%
\subsection{This is the softmax workbook for ECE C147/C247 Assignment
\#2}\label{this-is-the-softmax-workbook-for-ece-c147c247-assignment-2}}

Please follow the notebook linearly to implement a softmax classifier.

Please print out the workbook entirely when completed.

We thank Serena Yeung \& Justin Johnson for permission to use code
written for the CS 231n class (cs231n.stanford.edu). These are the
functions in the cs231n folders and code in the jupyer notebook to
preprocess and show the images. The classifiers used are based off of
code prepared for CS 231n as well.

The goal of this workbook is to give you experience with training a
softmax classifier.

    \begin{Verbatim}[commandchars=\\\{\}]
{\color{incolor}In [{\color{incolor}1}]:} \PY{k+kn}{import} \PY{n+nn}{random}
        \PY{k+kn}{import} \PY{n+nn}{numpy} \PY{k}{as} \PY{n+nn}{np}
        \PY{k+kn}{from} \PY{n+nn}{cs231n}\PY{n+nn}{.}\PY{n+nn}{data\PYZus{}utils} \PY{k}{import} \PY{n}{load\PYZus{}CIFAR10}
        \PY{k+kn}{import} \PY{n+nn}{matplotlib}\PY{n+nn}{.}\PY{n+nn}{pyplot} \PY{k}{as} \PY{n+nn}{plt}
        
        \PY{o}{\PYZpc{}}\PY{k}{matplotlib} inline
        \PY{o}{\PYZpc{}}\PY{k}{load\PYZus{}ext} autoreload
        \PY{o}{\PYZpc{}}\PY{k}{autoreload} 2
\end{Verbatim}


    \begin{Verbatim}[commandchars=\\\{\}]
{\color{incolor}In [{\color{incolor}2}]:} \PY{k}{def} \PY{n+nf}{get\PYZus{}CIFAR10\PYZus{}data}\PY{p}{(}\PY{n}{num\PYZus{}training}\PY{o}{=}\PY{l+m+mi}{49000}\PY{p}{,} \PY{n}{num\PYZus{}validation}\PY{o}{=}\PY{l+m+mi}{1000}\PY{p}{,} \PY{n}{num\PYZus{}test}\PY{o}{=}\PY{l+m+mi}{1000}\PY{p}{,} \PY{n}{num\PYZus{}dev}\PY{o}{=}\PY{l+m+mi}{500}\PY{p}{)}\PY{p}{:}
            \PY{l+s+sd}{\PYZdq{}\PYZdq{}\PYZdq{}}
        \PY{l+s+sd}{    Load the CIFAR\PYZhy{}10 dataset from disk and perform preprocessing to prepare}
        \PY{l+s+sd}{    it for the linear classifier. These are the same steps as we used for the}
        \PY{l+s+sd}{    SVM, but condensed to a single function.  }
        \PY{l+s+sd}{    \PYZdq{}\PYZdq{}\PYZdq{}}
            \PY{c+c1}{\PYZsh{} Load the raw CIFAR\PYZhy{}10 data}
            \PY{n}{cifar10\PYZus{}dir} \PY{o}{=} \PY{l+s+s1}{\PYZsq{}}\PY{l+s+s1}{./cifar\PYZhy{}10\PYZhy{}batches\PYZhy{}py}\PY{l+s+s1}{\PYZsq{}} \PY{c+c1}{\PYZsh{} You need to update this line}
            \PY{n}{X\PYZus{}train}\PY{p}{,} \PY{n}{y\PYZus{}train}\PY{p}{,} \PY{n}{X\PYZus{}test}\PY{p}{,} \PY{n}{y\PYZus{}test} \PY{o}{=} \PY{n}{load\PYZus{}CIFAR10}\PY{p}{(}\PY{n}{cifar10\PYZus{}dir}\PY{p}{)}
            
            \PY{c+c1}{\PYZsh{} subsample the data}
            \PY{n}{mask} \PY{o}{=} \PY{n+nb}{list}\PY{p}{(}\PY{n+nb}{range}\PY{p}{(}\PY{n}{num\PYZus{}training}\PY{p}{,} \PY{n}{num\PYZus{}training} \PY{o}{+} \PY{n}{num\PYZus{}validation}\PY{p}{)}\PY{p}{)}
            \PY{n}{X\PYZus{}val} \PY{o}{=} \PY{n}{X\PYZus{}train}\PY{p}{[}\PY{n}{mask}\PY{p}{]}
            \PY{n}{y\PYZus{}val} \PY{o}{=} \PY{n}{y\PYZus{}train}\PY{p}{[}\PY{n}{mask}\PY{p}{]}
            \PY{n}{mask} \PY{o}{=} \PY{n+nb}{list}\PY{p}{(}\PY{n+nb}{range}\PY{p}{(}\PY{n}{num\PYZus{}training}\PY{p}{)}\PY{p}{)}
            \PY{n}{X\PYZus{}train} \PY{o}{=} \PY{n}{X\PYZus{}train}\PY{p}{[}\PY{n}{mask}\PY{p}{]}
            \PY{n}{y\PYZus{}train} \PY{o}{=} \PY{n}{y\PYZus{}train}\PY{p}{[}\PY{n}{mask}\PY{p}{]}
            \PY{n}{mask} \PY{o}{=} \PY{n+nb}{list}\PY{p}{(}\PY{n+nb}{range}\PY{p}{(}\PY{n}{num\PYZus{}test}\PY{p}{)}\PY{p}{)}
            \PY{n}{X\PYZus{}test} \PY{o}{=} \PY{n}{X\PYZus{}test}\PY{p}{[}\PY{n}{mask}\PY{p}{]}
            \PY{n}{y\PYZus{}test} \PY{o}{=} \PY{n}{y\PYZus{}test}\PY{p}{[}\PY{n}{mask}\PY{p}{]}
            \PY{n}{mask} \PY{o}{=} \PY{n}{np}\PY{o}{.}\PY{n}{random}\PY{o}{.}\PY{n}{choice}\PY{p}{(}\PY{n}{num\PYZus{}training}\PY{p}{,} \PY{n}{num\PYZus{}dev}\PY{p}{,} \PY{n}{replace}\PY{o}{=}\PY{k+kc}{False}\PY{p}{)}
            \PY{n}{X\PYZus{}dev} \PY{o}{=} \PY{n}{X\PYZus{}train}\PY{p}{[}\PY{n}{mask}\PY{p}{]}
            \PY{n}{y\PYZus{}dev} \PY{o}{=} \PY{n}{y\PYZus{}train}\PY{p}{[}\PY{n}{mask}\PY{p}{]}
            
            \PY{c+c1}{\PYZsh{} Preprocessing: reshape the image data into rows}
            \PY{n}{X\PYZus{}train} \PY{o}{=} \PY{n}{np}\PY{o}{.}\PY{n}{reshape}\PY{p}{(}\PY{n}{X\PYZus{}train}\PY{p}{,} \PY{p}{(}\PY{n}{X\PYZus{}train}\PY{o}{.}\PY{n}{shape}\PY{p}{[}\PY{l+m+mi}{0}\PY{p}{]}\PY{p}{,} \PY{o}{\PYZhy{}}\PY{l+m+mi}{1}\PY{p}{)}\PY{p}{)}
            \PY{n}{X\PYZus{}val} \PY{o}{=} \PY{n}{np}\PY{o}{.}\PY{n}{reshape}\PY{p}{(}\PY{n}{X\PYZus{}val}\PY{p}{,} \PY{p}{(}\PY{n}{X\PYZus{}val}\PY{o}{.}\PY{n}{shape}\PY{p}{[}\PY{l+m+mi}{0}\PY{p}{]}\PY{p}{,} \PY{o}{\PYZhy{}}\PY{l+m+mi}{1}\PY{p}{)}\PY{p}{)}
            \PY{n}{X\PYZus{}test} \PY{o}{=} \PY{n}{np}\PY{o}{.}\PY{n}{reshape}\PY{p}{(}\PY{n}{X\PYZus{}test}\PY{p}{,} \PY{p}{(}\PY{n}{X\PYZus{}test}\PY{o}{.}\PY{n}{shape}\PY{p}{[}\PY{l+m+mi}{0}\PY{p}{]}\PY{p}{,} \PY{o}{\PYZhy{}}\PY{l+m+mi}{1}\PY{p}{)}\PY{p}{)}
            \PY{n}{X\PYZus{}dev} \PY{o}{=} \PY{n}{np}\PY{o}{.}\PY{n}{reshape}\PY{p}{(}\PY{n}{X\PYZus{}dev}\PY{p}{,} \PY{p}{(}\PY{n}{X\PYZus{}dev}\PY{o}{.}\PY{n}{shape}\PY{p}{[}\PY{l+m+mi}{0}\PY{p}{]}\PY{p}{,} \PY{o}{\PYZhy{}}\PY{l+m+mi}{1}\PY{p}{)}\PY{p}{)}
            
            \PY{c+c1}{\PYZsh{} Normalize the data: subtract the mean image}
            \PY{n}{mean\PYZus{}image} \PY{o}{=} \PY{n}{np}\PY{o}{.}\PY{n}{mean}\PY{p}{(}\PY{n}{X\PYZus{}train}\PY{p}{,} \PY{n}{axis} \PY{o}{=} \PY{l+m+mi}{0}\PY{p}{)}
            \PY{n}{X\PYZus{}train} \PY{o}{\PYZhy{}}\PY{o}{=} \PY{n}{mean\PYZus{}image}
            \PY{n}{X\PYZus{}val} \PY{o}{\PYZhy{}}\PY{o}{=} \PY{n}{mean\PYZus{}image}
            \PY{n}{X\PYZus{}test} \PY{o}{\PYZhy{}}\PY{o}{=} \PY{n}{mean\PYZus{}image}
            \PY{n}{X\PYZus{}dev} \PY{o}{\PYZhy{}}\PY{o}{=} \PY{n}{mean\PYZus{}image}
            
            \PY{c+c1}{\PYZsh{} add bias dimension and transform into columns}
            \PY{n}{X\PYZus{}train} \PY{o}{=} \PY{n}{np}\PY{o}{.}\PY{n}{hstack}\PY{p}{(}\PY{p}{[}\PY{n}{X\PYZus{}train}\PY{p}{,} \PY{n}{np}\PY{o}{.}\PY{n}{ones}\PY{p}{(}\PY{p}{(}\PY{n}{X\PYZus{}train}\PY{o}{.}\PY{n}{shape}\PY{p}{[}\PY{l+m+mi}{0}\PY{p}{]}\PY{p}{,} \PY{l+m+mi}{1}\PY{p}{)}\PY{p}{)}\PY{p}{]}\PY{p}{)}
            \PY{n}{X\PYZus{}val} \PY{o}{=} \PY{n}{np}\PY{o}{.}\PY{n}{hstack}\PY{p}{(}\PY{p}{[}\PY{n}{X\PYZus{}val}\PY{p}{,} \PY{n}{np}\PY{o}{.}\PY{n}{ones}\PY{p}{(}\PY{p}{(}\PY{n}{X\PYZus{}val}\PY{o}{.}\PY{n}{shape}\PY{p}{[}\PY{l+m+mi}{0}\PY{p}{]}\PY{p}{,} \PY{l+m+mi}{1}\PY{p}{)}\PY{p}{)}\PY{p}{]}\PY{p}{)}
            \PY{n}{X\PYZus{}test} \PY{o}{=} \PY{n}{np}\PY{o}{.}\PY{n}{hstack}\PY{p}{(}\PY{p}{[}\PY{n}{X\PYZus{}test}\PY{p}{,} \PY{n}{np}\PY{o}{.}\PY{n}{ones}\PY{p}{(}\PY{p}{(}\PY{n}{X\PYZus{}test}\PY{o}{.}\PY{n}{shape}\PY{p}{[}\PY{l+m+mi}{0}\PY{p}{]}\PY{p}{,} \PY{l+m+mi}{1}\PY{p}{)}\PY{p}{)}\PY{p}{]}\PY{p}{)}
            \PY{n}{X\PYZus{}dev} \PY{o}{=} \PY{n}{np}\PY{o}{.}\PY{n}{hstack}\PY{p}{(}\PY{p}{[}\PY{n}{X\PYZus{}dev}\PY{p}{,} \PY{n}{np}\PY{o}{.}\PY{n}{ones}\PY{p}{(}\PY{p}{(}\PY{n}{X\PYZus{}dev}\PY{o}{.}\PY{n}{shape}\PY{p}{[}\PY{l+m+mi}{0}\PY{p}{]}\PY{p}{,} \PY{l+m+mi}{1}\PY{p}{)}\PY{p}{)}\PY{p}{]}\PY{p}{)}
            
            \PY{k}{return} \PY{n}{X\PYZus{}train}\PY{p}{,} \PY{n}{y\PYZus{}train}\PY{p}{,} \PY{n}{X\PYZus{}val}\PY{p}{,} \PY{n}{y\PYZus{}val}\PY{p}{,} \PY{n}{X\PYZus{}test}\PY{p}{,} \PY{n}{y\PYZus{}test}\PY{p}{,} \PY{n}{X\PYZus{}dev}\PY{p}{,} \PY{n}{y\PYZus{}dev}
        
        
        \PY{c+c1}{\PYZsh{} Invoke the above function to get our data.}
        \PY{n}{X\PYZus{}train}\PY{p}{,} \PY{n}{y\PYZus{}train}\PY{p}{,} \PY{n}{X\PYZus{}val}\PY{p}{,} \PY{n}{y\PYZus{}val}\PY{p}{,} \PY{n}{X\PYZus{}test}\PY{p}{,} \PY{n}{y\PYZus{}test}\PY{p}{,} \PY{n}{X\PYZus{}dev}\PY{p}{,} \PY{n}{y\PYZus{}dev} \PY{o}{=} \PY{n}{get\PYZus{}CIFAR10\PYZus{}data}\PY{p}{(}\PY{p}{)}
        \PY{n+nb}{print}\PY{p}{(}\PY{l+s+s1}{\PYZsq{}}\PY{l+s+s1}{Train data shape: }\PY{l+s+s1}{\PYZsq{}}\PY{p}{,} \PY{n}{X\PYZus{}train}\PY{o}{.}\PY{n}{shape}\PY{p}{)}
        \PY{n+nb}{print}\PY{p}{(}\PY{l+s+s1}{\PYZsq{}}\PY{l+s+s1}{Train labels shape: }\PY{l+s+s1}{\PYZsq{}}\PY{p}{,} \PY{n}{y\PYZus{}train}\PY{o}{.}\PY{n}{shape}\PY{p}{)}
        \PY{n+nb}{print}\PY{p}{(}\PY{l+s+s1}{\PYZsq{}}\PY{l+s+s1}{Validation data shape: }\PY{l+s+s1}{\PYZsq{}}\PY{p}{,} \PY{n}{X\PYZus{}val}\PY{o}{.}\PY{n}{shape}\PY{p}{)}
        \PY{n+nb}{print}\PY{p}{(}\PY{l+s+s1}{\PYZsq{}}\PY{l+s+s1}{Validation labels shape: }\PY{l+s+s1}{\PYZsq{}}\PY{p}{,} \PY{n}{y\PYZus{}val}\PY{o}{.}\PY{n}{shape}\PY{p}{)}
        \PY{n+nb}{print}\PY{p}{(}\PY{l+s+s1}{\PYZsq{}}\PY{l+s+s1}{Test data shape: }\PY{l+s+s1}{\PYZsq{}}\PY{p}{,} \PY{n}{X\PYZus{}test}\PY{o}{.}\PY{n}{shape}\PY{p}{)}
        \PY{n+nb}{print}\PY{p}{(}\PY{l+s+s1}{\PYZsq{}}\PY{l+s+s1}{Test labels shape: }\PY{l+s+s1}{\PYZsq{}}\PY{p}{,} \PY{n}{y\PYZus{}test}\PY{o}{.}\PY{n}{shape}\PY{p}{)}
        \PY{n+nb}{print}\PY{p}{(}\PY{l+s+s1}{\PYZsq{}}\PY{l+s+s1}{dev data shape: }\PY{l+s+s1}{\PYZsq{}}\PY{p}{,} \PY{n}{X\PYZus{}dev}\PY{o}{.}\PY{n}{shape}\PY{p}{)}
        \PY{n+nb}{print}\PY{p}{(}\PY{l+s+s1}{\PYZsq{}}\PY{l+s+s1}{dev labels shape: }\PY{l+s+s1}{\PYZsq{}}\PY{p}{,} \PY{n}{y\PYZus{}dev}\PY{o}{.}\PY{n}{shape}\PY{p}{)}
\end{Verbatim}


    \begin{Verbatim}[commandchars=\\\{\}]
Train data shape:  (49000, 3073)
Train labels shape:  (49000,)
Validation data shape:  (1000, 3073)
Validation labels shape:  (1000,)
Test data shape:  (1000, 3073)
Test labels shape:  (1000,)
dev data shape:  (500, 3073)
dev labels shape:  (500,)

    \end{Verbatim}

    \hypertarget{training-a-softmax-classifier.}{%
\subsection{Training a softmax
classifier.}\label{training-a-softmax-classifier.}}

The following cells will take you through building a softmax classifier.
You will implement its loss function, then subsequently train it with
gradient descent. Finally, you will choose the learning rate of gradient
descent to optimize its classification performance.

    \begin{Verbatim}[commandchars=\\\{\}]
{\color{incolor}In [{\color{incolor}3}]:} \PY{k+kn}{from} \PY{n+nn}{nndl} \PY{k}{import} \PY{n}{Softmax}
\end{Verbatim}


    \begin{Verbatim}[commandchars=\\\{\}]
{\color{incolor}In [{\color{incolor}4}]:} \PY{c+c1}{\PYZsh{} Declare an instance of the Softmax class.  }
        \PY{c+c1}{\PYZsh{} Weights are initialized to a random value.}
        \PY{c+c1}{\PYZsh{} Note, to keep people\PYZsq{}s first solutions consistent, we are going to use a random seed.}
        
        \PY{n}{np}\PY{o}{.}\PY{n}{random}\PY{o}{.}\PY{n}{seed}\PY{p}{(}\PY{l+m+mi}{1}\PY{p}{)}
        
        \PY{n}{num\PYZus{}classes} \PY{o}{=} \PY{n+nb}{len}\PY{p}{(}\PY{n}{np}\PY{o}{.}\PY{n}{unique}\PY{p}{(}\PY{n}{y\PYZus{}train}\PY{p}{)}\PY{p}{)}
        \PY{n}{num\PYZus{}features} \PY{o}{=} \PY{n}{X\PYZus{}train}\PY{o}{.}\PY{n}{shape}\PY{p}{[}\PY{l+m+mi}{1}\PY{p}{]}
        
        \PY{n}{softmax} \PY{o}{=} \PY{n}{Softmax}\PY{p}{(}\PY{n}{dims}\PY{o}{=}\PY{p}{[}\PY{n}{num\PYZus{}classes}\PY{p}{,} \PY{n}{num\PYZus{}features}\PY{p}{]}\PY{p}{)}
\end{Verbatim}


    \hypertarget{softmax-loss}{%
\paragraph{Softmax loss}\label{softmax-loss}}

    \begin{Verbatim}[commandchars=\\\{\}]
{\color{incolor}In [{\color{incolor}5}]:} \PY{c+c1}{\PYZsh{}\PYZsh{} Implement the loss function of the softmax using a for loop over}
        \PY{c+c1}{\PYZsh{}  the number of examples}
        
        \PY{n}{loss} \PY{o}{=} \PY{n}{softmax}\PY{o}{.}\PY{n}{loss}\PY{p}{(}\PY{n}{X\PYZus{}train}\PY{p}{,} \PY{n}{y\PYZus{}train}\PY{p}{)}
\end{Verbatim}


    \begin{Verbatim}[commandchars=\\\{\}]
{\color{incolor}In [{\color{incolor}6}]:} \PY{n+nb}{print}\PY{p}{(}\PY{n}{loss}\PY{p}{)}
\end{Verbatim}


    \begin{Verbatim}[commandchars=\\\{\}]
2.3277607028048966

    \end{Verbatim}

    \hypertarget{question}{%
\subsection{Question:}\label{question}}

You'll notice the loss returned by the softmax is about 2.3 (if
implemented correctly). Why does this make sense?

    \hypertarget{answer}{%
\subsection{Answer:}\label{answer}}

Since our weights are random, we expect our model to guess the labels
correctly 1/10 of the time. That means that the proportion in the
softmax function will be \textasciitilde1/10. Our loss function is
\(-log (softmax)\), so we get \(-log (\frac{1}{10}) \approx 2.3\)

    \hypertarget{softmax-gradient}{%
\paragraph{Softmax gradient}\label{softmax-gradient}}

    \begin{Verbatim}[commandchars=\\\{\}]
{\color{incolor}In [{\color{incolor}7}]:} \PY{c+c1}{\PYZsh{}\PYZsh{} Calculate the gradient of the softmax loss in the Softmax class.}
        \PY{c+c1}{\PYZsh{} For convenience, we\PYZsq{}ll write one function that computes the loss}
        \PY{c+c1}{\PYZsh{}   and gradient together, softmax.loss\PYZus{}and\PYZus{}grad(X, y)}
        \PY{c+c1}{\PYZsh{} You may copy and paste your loss code from softmax.loss() here, and then}
        \PY{c+c1}{\PYZsh{}   use the appropriate intermediate values to calculate the gradient.}
        
        \PY{n}{loss}\PY{p}{,} \PY{n}{grad} \PY{o}{=} \PY{n}{softmax}\PY{o}{.}\PY{n}{loss\PYZus{}and\PYZus{}grad}\PY{p}{(}\PY{n}{X\PYZus{}dev}\PY{p}{,}\PY{n}{y\PYZus{}dev}\PY{p}{)}
        
        \PY{c+c1}{\PYZsh{} Compare your gradient to a gradient check we wrote. }
        \PY{c+c1}{\PYZsh{} You should see relative gradient errors on the order of 1e\PYZhy{}07 or less if you implemented the gradient correctly.}
        \PY{n}{softmax}\PY{o}{.}\PY{n}{grad\PYZus{}check\PYZus{}sparse}\PY{p}{(}\PY{n}{X\PYZus{}dev}\PY{p}{,} \PY{n}{y\PYZus{}dev}\PY{p}{,} \PY{n}{grad}\PY{p}{)}
\end{Verbatim}


    \begin{Verbatim}[commandchars=\\\{\}]
numerical: 0.332009 analytic: 0.332009, relative error: 8.056897e-08
numerical: 0.543674 analytic: 0.543674, relative error: 3.675417e-08
numerical: -0.834213 analytic: -0.834213, relative error: 3.539436e-08
numerical: 2.460479 analytic: 2.460479, relative error: 5.566823e-09
numerical: -0.788769 analytic: -0.788769, relative error: 1.392953e-08
numerical: 0.832805 analytic: 0.832805, relative error: 2.331392e-08
numerical: 0.044043 analytic: 0.044043, relative error: 9.778745e-07
numerical: -1.186865 analytic: -1.186865, relative error: 2.435639e-09
numerical: 1.510502 analytic: 1.510502, relative error: 4.114666e-09
numerical: -2.932045 analytic: -2.932046, relative error: 1.727180e-08

    \end{Verbatim}

    \hypertarget{a-vectorized-version-of-softmax}{%
\subsection{A vectorized version of
Softmax}\label{a-vectorized-version-of-softmax}}

To speed things up, we will vectorize the loss and gradient
calculations. This will be helpful for stochastic gradient descent.

    \begin{Verbatim}[commandchars=\\\{\}]
{\color{incolor}In [{\color{incolor}8}]:} \PY{k+kn}{import} \PY{n+nn}{time}
\end{Verbatim}


    \begin{Verbatim}[commandchars=\\\{\}]
{\color{incolor}In [{\color{incolor}9}]:} \PY{c+c1}{\PYZsh{}\PYZsh{} Implement softmax.fast\PYZus{}loss\PYZus{}and\PYZus{}grad which calculates the loss and gradient}
        \PY{c+c1}{\PYZsh{}    WITHOUT using any for loops.  }
        
        \PY{c+c1}{\PYZsh{} Standard loss and gradient}
        \PY{n}{tic} \PY{o}{=} \PY{n}{time}\PY{o}{.}\PY{n}{time}\PY{p}{(}\PY{p}{)}
        \PY{n}{loss}\PY{p}{,} \PY{n}{grad} \PY{o}{=} \PY{n}{softmax}\PY{o}{.}\PY{n}{loss\PYZus{}and\PYZus{}grad}\PY{p}{(}\PY{n}{X\PYZus{}dev}\PY{p}{,} \PY{n}{y\PYZus{}dev}\PY{p}{)}
        \PY{n}{toc} \PY{o}{=} \PY{n}{time}\PY{o}{.}\PY{n}{time}\PY{p}{(}\PY{p}{)}
        \PY{n+nb}{print}\PY{p}{(}\PY{l+s+s1}{\PYZsq{}}\PY{l+s+s1}{Normal loss / grad\PYZus{}norm: }\PY{l+s+si}{\PYZob{}\PYZcb{}}\PY{l+s+s1}{ / }\PY{l+s+si}{\PYZob{}\PYZcb{}}\PY{l+s+s1}{ computed in }\PY{l+s+si}{\PYZob{}\PYZcb{}}\PY{l+s+s1}{s}\PY{l+s+s1}{\PYZsq{}}\PY{o}{.}\PY{n}{format}\PY{p}{(}\PY{n}{loss}\PY{p}{,} \PY{n}{np}\PY{o}{.}\PY{n}{linalg}\PY{o}{.}\PY{n}{norm}\PY{p}{(}\PY{n}{grad}\PY{p}{,} \PY{l+s+s1}{\PYZsq{}}\PY{l+s+s1}{fro}\PY{l+s+s1}{\PYZsq{}}\PY{p}{)}\PY{p}{,} \PY{n}{toc} \PY{o}{\PYZhy{}} \PY{n}{tic}\PY{p}{)}\PY{p}{)}
        
        \PY{n}{tic} \PY{o}{=} \PY{n}{time}\PY{o}{.}\PY{n}{time}\PY{p}{(}\PY{p}{)}
        \PY{n}{loss\PYZus{}vectorized}\PY{p}{,} \PY{n}{grad\PYZus{}vectorized} \PY{o}{=} \PY{n}{softmax}\PY{o}{.}\PY{n}{fast\PYZus{}loss\PYZus{}and\PYZus{}grad}\PY{p}{(}\PY{n}{X\PYZus{}dev}\PY{p}{,} \PY{n}{y\PYZus{}dev}\PY{p}{)}
        \PY{n}{toc} \PY{o}{=} \PY{n}{time}\PY{o}{.}\PY{n}{time}\PY{p}{(}\PY{p}{)}
        \PY{n+nb}{print}\PY{p}{(}\PY{l+s+s1}{\PYZsq{}}\PY{l+s+s1}{Vectorized loss / grad: }\PY{l+s+si}{\PYZob{}\PYZcb{}}\PY{l+s+s1}{ / }\PY{l+s+si}{\PYZob{}\PYZcb{}}\PY{l+s+s1}{ computed in }\PY{l+s+si}{\PYZob{}\PYZcb{}}\PY{l+s+s1}{s}\PY{l+s+s1}{\PYZsq{}}\PY{o}{.}\PY{n}{format}\PY{p}{(}\PY{n}{loss\PYZus{}vectorized}\PY{p}{,} \PY{n}{np}\PY{o}{.}\PY{n}{linalg}\PY{o}{.}\PY{n}{norm}\PY{p}{(}\PY{n}{grad\PYZus{}vectorized}\PY{p}{,} \PY{l+s+s1}{\PYZsq{}}\PY{l+s+s1}{fro}\PY{l+s+s1}{\PYZsq{}}\PY{p}{)}\PY{p}{,} \PY{n}{toc} \PY{o}{\PYZhy{}} \PY{n}{tic}\PY{p}{)}\PY{p}{)}
        
        \PY{c+c1}{\PYZsh{} The losses should match but your vectorized implementation should be much faster.}
        \PY{n+nb}{print}\PY{p}{(}\PY{l+s+s1}{\PYZsq{}}\PY{l+s+s1}{difference in loss / grad: }\PY{l+s+si}{\PYZob{}\PYZcb{}}\PY{l+s+s1}{ /}\PY{l+s+si}{\PYZob{}\PYZcb{}}\PY{l+s+s1}{ }\PY{l+s+s1}{\PYZsq{}}\PY{o}{.}\PY{n}{format}\PY{p}{(}\PY{n}{loss} \PY{o}{\PYZhy{}} \PY{n}{loss\PYZus{}vectorized}\PY{p}{,} \PY{n}{np}\PY{o}{.}\PY{n}{linalg}\PY{o}{.}\PY{n}{norm}\PY{p}{(}\PY{n}{grad} \PY{o}{\PYZhy{}} \PY{n}{grad\PYZus{}vectorized}\PY{p}{)}\PY{p}{)}\PY{p}{)}
        
        \PY{c+c1}{\PYZsh{} You should notice a speedup with the same output.}
\end{Verbatim}


    \begin{Verbatim}[commandchars=\\\{\}]
Normal loss / grad\_norm: 2.350458377628373 / 312.42133095774625 computed in 0.2377328872680664s
Vectorized loss / grad: 2.3504583776283714 / 312.4213309577462 computed in 0.014288187026977539s
difference in loss / grad: 1.7763568394002505e-15 /2.8663620561962255e-13 

    \end{Verbatim}

    \hypertarget{stochastic-gradient-descent}{%
\subsection{Stochastic gradient
descent}\label{stochastic-gradient-descent}}

We now implement stochastic gradient descent. This uses the same
principles of gradient descent we discussed in class, however, it
calculates the gradient by only using examples from a subset of the
training set (so each gradient calculation is faster).

    \hypertarget{question}{%
\subsection{Question:}\label{question}}

How should the softmax gradient descent training step differ from the
svm training step, if at all?

    \hypertarget{answer}{%
\subsection{Answer:}\label{answer}}

The softmax gradient descent training step should be identical to the
svm training step.

    \begin{Verbatim}[commandchars=\\\{\}]
{\color{incolor}In [{\color{incolor}10}]:} \PY{c+c1}{\PYZsh{} Implement softmax.train() by filling in the code to extract a batch of data}
         \PY{c+c1}{\PYZsh{} and perform the gradient step.}
         \PY{k+kn}{import} \PY{n+nn}{time}
         
         
         \PY{n}{tic} \PY{o}{=} \PY{n}{time}\PY{o}{.}\PY{n}{time}\PY{p}{(}\PY{p}{)}
         \PY{n}{loss\PYZus{}hist} \PY{o}{=} \PY{n}{softmax}\PY{o}{.}\PY{n}{train}\PY{p}{(}\PY{n}{X\PYZus{}train}\PY{p}{,} \PY{n}{y\PYZus{}train}\PY{p}{,} \PY{n}{learning\PYZus{}rate}\PY{o}{=}\PY{l+m+mf}{1e\PYZhy{}7}\PY{p}{,}
                               \PY{n}{num\PYZus{}iters}\PY{o}{=}\PY{l+m+mi}{1500}\PY{p}{,} \PY{n}{verbose}\PY{o}{=}\PY{k+kc}{True}\PY{p}{)}
         \PY{n}{toc} \PY{o}{=} \PY{n}{time}\PY{o}{.}\PY{n}{time}\PY{p}{(}\PY{p}{)}
         \PY{n+nb}{print}\PY{p}{(}\PY{l+s+s1}{\PYZsq{}}\PY{l+s+s1}{That took }\PY{l+s+si}{\PYZob{}\PYZcb{}}\PY{l+s+s1}{s}\PY{l+s+s1}{\PYZsq{}}\PY{o}{.}\PY{n}{format}\PY{p}{(}\PY{n}{toc} \PY{o}{\PYZhy{}} \PY{n}{tic}\PY{p}{)}\PY{p}{)}
         
         \PY{n}{plt}\PY{o}{.}\PY{n}{plot}\PY{p}{(}\PY{n}{loss\PYZus{}hist}\PY{p}{)}
         \PY{n}{plt}\PY{o}{.}\PY{n}{xlabel}\PY{p}{(}\PY{l+s+s1}{\PYZsq{}}\PY{l+s+s1}{Iteration number}\PY{l+s+s1}{\PYZsq{}}\PY{p}{)}
         \PY{n}{plt}\PY{o}{.}\PY{n}{ylabel}\PY{p}{(}\PY{l+s+s1}{\PYZsq{}}\PY{l+s+s1}{Loss value}\PY{l+s+s1}{\PYZsq{}}\PY{p}{)}
         \PY{n}{plt}\PY{o}{.}\PY{n}{show}\PY{p}{(}\PY{p}{)}
\end{Verbatim}


    \begin{Verbatim}[commandchars=\\\{\}]
iteration 0 / 1500: loss 2.3365926606637544
iteration 100 / 1500: loss 2.0557222613850827
iteration 200 / 1500: loss 2.0357745120662813
iteration 300 / 1500: loss 1.9813348165609888
iteration 400 / 1500: loss 1.9583142443981612
iteration 500 / 1500: loss 1.862265307354135
iteration 600 / 1500: loss 1.8532611454359382
iteration 700 / 1500: loss 1.8353062223725827
iteration 800 / 1500: loss 1.829389246882764
iteration 900 / 1500: loss 1.899215853035748
iteration 1000 / 1500: loss 1.97835035402523
iteration 1100 / 1500: loss 1.8470797913532633
iteration 1200 / 1500: loss 1.8411450268664082
iteration 1300 / 1500: loss 1.7910402495792102
iteration 1400 / 1500: loss 1.8705803029382257
That took 15.245999097824097s

    \end{Verbatim}

    \begin{center}
    \adjustimage{max size={0.9\linewidth}{0.9\paperheight}}{output_19_1.png}
    \end{center}
    { \hspace*{\fill} \\}
    
    \hypertarget{evaluate-the-performance-of-the-trained-softmax-classifier-on-the-validation-data.}{%
\subsubsection{Evaluate the performance of the trained softmax
classifier on the validation
data.}\label{evaluate-the-performance-of-the-trained-softmax-classifier-on-the-validation-data.}}

    \begin{Verbatim}[commandchars=\\\{\}]
{\color{incolor}In [{\color{incolor}11}]:} \PY{c+c1}{\PYZsh{}\PYZsh{} Implement softmax.predict() and use it to compute the training and testing error.}
         
         \PY{n}{y\PYZus{}train\PYZus{}pred} \PY{o}{=} \PY{n}{softmax}\PY{o}{.}\PY{n}{predict}\PY{p}{(}\PY{n}{X\PYZus{}train}\PY{p}{)}
         \PY{n+nb}{print}\PY{p}{(}\PY{l+s+s1}{\PYZsq{}}\PY{l+s+s1}{training accuracy: }\PY{l+s+si}{\PYZob{}\PYZcb{}}\PY{l+s+s1}{\PYZsq{}}\PY{o}{.}\PY{n}{format}\PY{p}{(}\PY{n}{np}\PY{o}{.}\PY{n}{mean}\PY{p}{(}\PY{n}{np}\PY{o}{.}\PY{n}{equal}\PY{p}{(}\PY{n}{y\PYZus{}train}\PY{p}{,}\PY{n}{y\PYZus{}train\PYZus{}pred}\PY{p}{)}\PY{p}{,} \PY{p}{)}\PY{p}{)}\PY{p}{)}
         \PY{n}{y\PYZus{}val\PYZus{}pred} \PY{o}{=} \PY{n}{softmax}\PY{o}{.}\PY{n}{predict}\PY{p}{(}\PY{n}{X\PYZus{}val}\PY{p}{)}
         \PY{n+nb}{print}\PY{p}{(}\PY{l+s+s1}{\PYZsq{}}\PY{l+s+s1}{validation accuracy: }\PY{l+s+si}{\PYZob{}\PYZcb{}}\PY{l+s+s1}{\PYZsq{}}\PY{o}{.}\PY{n}{format}\PY{p}{(}\PY{n}{np}\PY{o}{.}\PY{n}{mean}\PY{p}{(}\PY{n}{np}\PY{o}{.}\PY{n}{equal}\PY{p}{(}\PY{n}{y\PYZus{}val}\PY{p}{,} \PY{n}{y\PYZus{}val\PYZus{}pred}\PY{p}{)}\PY{p}{)}\PY{p}{,} \PY{p}{)}\PY{p}{)}
\end{Verbatim}


    \begin{Verbatim}[commandchars=\\\{\}]
training accuracy: 0.3811428571428571
validation accuracy: 0.398

    \end{Verbatim}

    \hypertarget{optimize-the-softmax-classifier}{%
\subsection{Optimize the softmax
classifier}\label{optimize-the-softmax-classifier}}

You may copy and paste your optimization code from the SVM here.

    \begin{Verbatim}[commandchars=\\\{\}]
{\color{incolor}In [{\color{incolor}12}]:} \PY{n}{np}\PY{o}{.}\PY{n}{finfo}\PY{p}{(}\PY{n+nb}{float}\PY{p}{)}\PY{o}{.}\PY{n}{eps}
\end{Verbatim}


\begin{Verbatim}[commandchars=\\\{\}]
{\color{outcolor}Out[{\color{outcolor}12}]:} 2.220446049250313e-16
\end{Verbatim}
            
    \begin{Verbatim}[commandchars=\\\{\}]
{\color{incolor}In [{\color{incolor}13}]:} \PY{c+c1}{\PYZsh{} ================================================================ \PYZsh{}}
         \PY{c+c1}{\PYZsh{} YOUR CODE HERE:}
         \PY{c+c1}{\PYZsh{}   Train the Softmax classifier with different learning rates and }
         \PY{c+c1}{\PYZsh{}     evaluate on the validation data.}
         \PY{c+c1}{\PYZsh{}   Report:}
         \PY{c+c1}{\PYZsh{}     \PYZhy{} The best learning rate of the ones you tested.  }
         \PY{c+c1}{\PYZsh{}     \PYZhy{} The best validation accuracy corresponding to the best validation error.}
         \PY{c+c1}{\PYZsh{}}
         \PY{c+c1}{\PYZsh{}   Select the SVM that achieved the best validation error and report}
         \PY{c+c1}{\PYZsh{}     its error rate on the test set.}
         \PY{c+c1}{\PYZsh{} ================================================================ \PYZsh{}}
         
         \PY{n}{learning\PYZus{}rates} \PY{o}{=} \PY{p}{[}\PY{l+m+mf}{1e\PYZhy{}4}\PY{p}{,} \PY{l+m+mf}{1e\PYZhy{}3}\PY{p}{,} \PY{l+m+mf}{1e\PYZhy{}2}\PY{p}{,} \PY{l+m+mf}{5e\PYZhy{}2}\PY{p}{,} \PY{l+m+mf}{0.1}\PY{p}{,} \PY{l+m+mf}{0.25}\PY{p}{,} \PY{l+m+mf}{0.5}\PY{p}{]}
         \PY{n}{val\PYZus{}accs} \PY{o}{=} \PY{n}{np}\PY{o}{.}\PY{n}{zeros}\PY{p}{(}\PY{n+nb}{len}\PY{p}{(}\PY{n}{learning\PYZus{}rates}\PY{p}{)}\PY{p}{)}
         \PY{k}{for} \PY{n}{i} \PY{o+ow}{in} \PY{n+nb}{range}\PY{p}{(}\PY{n+nb}{len}\PY{p}{(}\PY{n}{learning\PYZus{}rates}\PY{p}{)}\PY{p}{)}\PY{p}{:}
             \PY{n}{softmax}\PY{o}{.}\PY{n}{train}\PY{p}{(}\PY{n}{X\PYZus{}train}\PY{p}{,} \PY{n}{y\PYZus{}train}\PY{p}{,} \PY{n}{learning\PYZus{}rate}\PY{o}{=}\PY{n}{learning\PYZus{}rates}\PY{p}{[}\PY{n}{i}\PY{p}{]}\PY{p}{,} \PY{n}{num\PYZus{}iters}\PY{o}{=}\PY{l+m+mi}{1500}\PY{p}{,} \PY{n}{verbose}\PY{o}{=}\PY{k+kc}{False}\PY{p}{)}
             \PY{n}{val\PYZus{}accs}\PY{p}{[}\PY{n}{i}\PY{p}{]} \PY{o}{=} \PY{n}{np}\PY{o}{.}\PY{n}{mean}\PY{p}{(}\PY{n}{np}\PY{o}{.}\PY{n}{equal}\PY{p}{(}\PY{n}{y\PYZus{}val}\PY{p}{,} \PY{n}{softmax}\PY{o}{.}\PY{n}{predict}\PY{p}{(}\PY{n}{X\PYZus{}val}\PY{p}{)}\PY{p}{)}\PY{p}{)}
         
         \PY{n}{best\PYZus{}learning\PYZus{}rate} \PY{o}{=} \PY{n}{learning\PYZus{}rates}\PY{p}{[}\PY{n}{np}\PY{o}{.}\PY{n}{argmax}\PY{p}{(}\PY{n}{val\PYZus{}accs}\PY{p}{)}\PY{p}{]}
         \PY{n}{best\PYZus{}val\PYZus{}acc} \PY{o}{=} \PY{n}{np}\PY{o}{.}\PY{n}{max}\PY{p}{(}\PY{n}{val\PYZus{}accs}\PY{p}{)}
         
         \PY{n+nb}{print}\PY{p}{(}\PY{l+s+s1}{\PYZsq{}}\PY{l+s+s1}{best learning rate: }\PY{l+s+si}{\PYZob{}\PYZcb{}}\PY{l+s+s1}{\PYZsq{}}\PY{o}{.}\PY{n}{format}\PY{p}{(}\PY{n}{best\PYZus{}learning\PYZus{}rate}\PY{p}{)}\PY{p}{)}
         \PY{n+nb}{print}\PY{p}{(}\PY{l+s+s1}{\PYZsq{}}\PY{l+s+s1}{best validation accuracy: }\PY{l+s+si}{\PYZob{}\PYZcb{}}\PY{l+s+s1}{\PYZsq{}}\PY{o}{.}\PY{n}{format}\PY{p}{(}\PY{n}{best\PYZus{}val\PYZus{}acc}\PY{p}{)}\PY{p}{)}
         
         \PY{n}{softmax}\PY{o}{.}\PY{n}{train}\PY{p}{(}\PY{n}{X\PYZus{}train}\PY{p}{,} \PY{n}{y\PYZus{}train}\PY{p}{,} \PY{n}{learning\PYZus{}rate}\PY{o}{=}\PY{n}{best\PYZus{}learning\PYZus{}rate}\PY{p}{,} \PY{n}{num\PYZus{}iters}\PY{o}{=}\PY{l+m+mi}{1500}\PY{p}{,} \PY{n}{verbose}\PY{o}{=}\PY{k+kc}{False}\PY{p}{)}
         \PY{n}{test\PYZus{}pred} \PY{o}{=} \PY{n}{softmax}\PY{o}{.}\PY{n}{predict}\PY{p}{(}\PY{n}{X\PYZus{}test}\PY{p}{)}
         \PY{n}{test\PYZus{}error} \PY{o}{=} \PY{l+m+mi}{1} \PY{o}{\PYZhy{}} \PY{n}{np}\PY{o}{.}\PY{n}{mean}\PY{p}{(}\PY{n}{np}\PY{o}{.}\PY{n}{equal}\PY{p}{(}\PY{n}{y\PYZus{}test}\PY{p}{,} \PY{n}{test\PYZus{}pred}\PY{p}{)}\PY{p}{)}
         
         \PY{n+nb}{print}\PY{p}{(}\PY{l+s+s1}{\PYZsq{}}\PY{l+s+s1}{final test error rate: }\PY{l+s+si}{\PYZob{}\PYZcb{}}\PY{l+s+s1}{\PYZsq{}}\PY{o}{.}\PY{n}{format}\PY{p}{(}\PY{n}{test\PYZus{}error}\PY{p}{)}\PY{p}{)}
         
         \PY{c+c1}{\PYZsh{} ================================================================ \PYZsh{}}
         \PY{c+c1}{\PYZsh{} END YOUR CODE HERE}
         \PY{c+c1}{\PYZsh{} ================================================================ \PYZsh{}}
\end{Verbatim}


    \begin{Verbatim}[commandchars=\\\{\}]
/Users/edwardzhang/Desktop/ece247/HW2/HW2-code/nndl/softmax.py:131: RuntimeWarning: divide by zero encountered in log
  loss = np.sum(np.log(np.sum(np.exp(a).T, axis=0)) - a[np.arange(num\_train), y]) / num\_train
/Users/edwardzhang/Desktop/ece247/HW2/HW2-code/nndl/softmax.py:135: RuntimeWarning: invalid value encountered in true\_divide
  softmax = ea / sums[:, np.newaxis]

    \end{Verbatim}

    \begin{Verbatim}[commandchars=\\\{\}]
best learning rate: 0.0001
best validation accuracy: 0.297
final test error rate: 0.6719999999999999

    \end{Verbatim}


    % Add a bibliography block to the postdoc
    
    
    
    \end{document}
